\documentclass[12pt]{IEEEtran} % .... using IEEE Transaction style
\usepackage{graphics}

% the following is to get inch margins on a letter-size paper
\setlength{\topmargin}{0pt}
\setlength{\headheight}{0in}
\setlength{\headsep}{0in}
\setlength{\textheight}{9.0in}
\setlength{\footskip}{0.5in}
\setlength{\oddsidemargin}{0pt}
\setlength{\evensidemargin}{0pt}
\setlength{\textwidth}{6.5in}

\begin{document}
\title{A Java Wrapper For Embarresingly Parallel Programs}
\author{Jacob Schwartz}
\maketitle

\begin{abstract}
Lorem ipsum dolor sit amet, consectetur adipisicing elit, sed do eiusmod
tempor incididunt ut labore et dolore magna aliqua. Ut enim ad minim
veniam, quis nostrud exercitation ullamco laboris nisi ut aliquip ex ea
commodo consequat. Duis aute irure dolor in reprehenderit in voluptate
velit esse cillum dolore eu fugiat nulla pariatur. Excepteur sint occaecat
cupidatat non proident, sunt in culpa qui officia deserunt mollit anim
id est laborum.
\end{abstract}


\section{Introduction}

This is just a test. This is just a test (a line break follows)\\
this is just a test.


\subsection*{Background}

Figure \ref{fig:figure} shows .... Has to be compiled {\bf twice} to get the 
figure reference right. AODV \cite{aodv} is an ad hoc roting protocol.

Lorem ipsum dolor sit amet, consectetur adipisicing elit, sed do eiusmod
tempor incididunt ut labore et dolore magna aliqua. Ut enim ad minim
veniam, quis nostrud exercitation ullamco laboris nisi ut aliquip ex ea
commodo consequat. Duis aute irure dolor in reprehenderit in voluptate
velit esse cillum dolore eu fugiat nulla pariatur. Excepteur sint occaecat
cupidatat non proident, sunt in culpa qui officia deserunt mollit anim
id est laborum.

\begin{figure}[bht]
\begin{center}
{\resizebox{2in}{!}{\includegraphics{figures/drawing}}}
%        {x-dim}{y-dim}
% ! means proportional scaling
\end{center}
\caption{Caption of the figure.}
\label{fig:figure}
\end{figure}

\begin{itemize}
\item One
\item Two
\item There
\end{itemize}

\begin{enumerate}
\item One
\item Two
\item There
\end{enumerate}

\begin{description}
\item[Item 1:] Ut enim ad minim veniam, quis nostrud exercitation ullamco
laboris nisi ut aliquip ex ea commodo consequat.
\item[Item 2:]  Duis aute irure dolor in reprehenderit in voluptate
velit esse cillum dolore eu fugiat nulla pariatur.
\item[Item 3:]  Excepteur sint occaecat cupidatat non proident, sunt in
culpa qui officia deserunt mollit anim id est laborum.
\end{description}

\begin{table}[htdp]
\caption{default}
\begin{center}
\begin{tabular}{|c|c|}
\hline
jljljljl&jkkjk\\
\hline
\end{tabular}
\end{center}
\label{default}
\end{table}%


\section{Experiment}

The results of experiments are 

$$\frac{3}{2}$$ 

shown in Figure~\ref{fig:graph}.

\begin{figure}[bht]
\begin{center}
%{\resizebox{3.4in}{!}{\includegraphics{experiment/plot}}}
%        {x-dim}{y-dim}
% ! means proportional scaling
\end{center}
\caption{Caption of the graph.}
\label{fig:graph}
\end{figure}


\begin{thebibliography}{1}

\bibitem{aodv}
C. Perkins,  E. Belding-Royer, and  S. Das, ``\emph{Ad hoc On-Demand
Distance Vector (AODV) Routing},'' IETF RFC 3561, July 2003.

\end{thebibliography}

\end{document}
