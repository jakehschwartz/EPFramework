\documentclass[12pt]{article} 
\usepackage{graphics}
\usepackage{setspace}

% the following is to get inch margins on a letter-size paper
\setlength{\topmargin}{0pt}
\setlength{\headheight}{0in}
\setlength{\headsep}{0in}
\setlength{\textheight}{9.0in}
\setlength{\footskip}{0.5in}
\setlength{\oddsidemargin}{0pt}
\setlength{\evensidemargin}{0pt}
\setlength{\textwidth}{6.5in}

\doublespacing
\begin{document}
\title{A Java Wrapper For Embarresingly Parallel Problems}
\author{Jacob Schwartz}
\maketitle

\begin{abstract}
An embarresingly parallel, also called pleasing parallel, problem is one that 
can easily be broken up into BLANK that do not need to know about one another.
Several problems in the biological sciences are embarrsingly parallel but they
are difficult to rewrite to use multithreading, because they are either poorly
written or too complex. We have written a wrapper, implemented in Java, that
will execute the serial program in several threads. RESULTS SENTENCE. More
statistics, savable configurations and BLANK will be implemented next, in
addition to BLANK.
\end{abstract}


\section{Introduction}

HISTORY OF PARALLELISM

DISCUSS EMBARRESINGKLY PARALLEL (GET 3 LINKS BEFORE HAND)

TALK ABOUT BIOLOGICAL SCIENCES AND KAKS?

WHAT ARE WE GOING TO DO

SUMMARY OF RESULTS

\section{Background}

BACKGROUND ON WORK WITH DAVE. INTRO TO KAKS.

Instead of attempting to rewrite the preexisting program, blah. Trying to
parallelize the serial KaKs Calculator program would have presented several
problems: introducing bugs, breaking the KaKs algorithm, and the learning curve
of C++ threading. With such a short time before results were needed, a different
approach was necessary. A simple Java program was written to break up the input
and spawn threads that would run BLANKS's KaKs Calculator. When the input was
calculated, the contents of the output files was concatenated together to
produce a final input. MORE?

The KaKs Calculator is not the only embarresingly parallel program that has the
potential to take several days to run on a data set. In fact, there are several
embarresingly parallel programs in the biological sciences. TALK ABOUT OTHER
PROGRAMS HERE. With this multitude of applications that would benefit from
parallelism, it was obvious that the wrapper program has potential outside of
this project.

TALK ABOUT PAPER HERE. AND MAYBE THE POWERPOINT TOO.

The original wrapper was also written very quickly due to the impending
deadline. In this version of the wrapper, several components are being added.
First off, there is a wizard style menu that will allow the end-user to select
not only the program that wish to use, but also to manage other settings for the
run. A new IO class splits the input and merges the output more 
efficiently. Lastly, in order to look at how well the wrapper performs, 
the threads keep their own statstics. These statstics will not only be used to 
find potential enhancements but also will help end-users of the wrapper
determine whether their use of the wrapper is efficient.

\section{Proposed Solution}

Break down the components. Have a UML diagram here first

The KaKs Calculator is not the only program to used the wrapper; there are
various potential programs that the wrapper could support. SENTANCE. The wrapper
needs to support various kinds of input files and be able to chunk up the work
accordingly. The KaKs Calculator's input consists of a header and then two
protein sequence lines followed by a blank line. The wrapper will use that blank
line to know when one chunk of work has ended and another has begun. The user
can input their own regular expression to determine what lines of input make up
what chunks. This will occur in the ChunkManager class. 

The other purpose of the ChunkManager class is to combine the resulting chunks 
back together when the worker threads have finished. MORE

As more features are added to the wrapper and as more programs start to use the
wrapper, there are more choices for the end-user to make in regards to how they
want their program to execute. The number of threads to use, various flags for
the executable and blank are just a few choice they can make. A start-up wizard
is displayed at runtime to systimatically allow the end-user to choose the
settings for the run. The ConfigWizard class brings the user through this
process and creates an instance of Configuration when it is completed to save
the user's settings.

Lastly, the worker threads will keep records during the execution of a set of 
work. The threads records its uptime and the number of pieces of work it
executes on. Also, the runtime for each piece of work is recorded. These will be
used by the end user to determine if their run is ideal or if they may need to
change some settings to make it work faster next time. The statistics can also
give the developers of the wrapper an idea on what can be implemented in order
make it as fast as possible for a user. The statistics for each thread can be
found in the Worker subclass and the individual Chunk classes will store their
own statstics about runtime and size.

\section{Results}

Tables charts, etc

\section{Conclusion}

Conclusions gathered from results.

There are several next steps in the pipeline. In addition to using the
statistics to try to boost speed, adding savable configurations is at the top of
the list. This will allow users to use the same configuration without having to
go through the wizard if they need to redo a run. More extensive unit testing
will be implemented. The ability to chunk multiple pieces of work may be
implemented in order to avoid potential start up costs in the executable. 
Finally, other  pleasing parallel problems will be executed by the wrapper so 
that more potential features will be discovered. MORE

Using Java was a good choice for implementation ease and future implementation
extension by others, but it may not produce the fastest results. Other languages
like Hadoop or blank may have been better due to their parallel nature. DO SOME
RESEARCH. Another approach that could have been taken is one similar to the
bitches in Washington: going to the cloud. WHY THEY WOULD BE BETTER. HOW IT
COULD BE IMPLEMENTED.



\begin{thebibliography}{1}

\bibitem{aodv}
C. Perkins,  E. Belding-Royer, and  S. Das, ``\emph{Ad hoc On-Demand
Distance Vector (AODV) Routing},'' IETF RFC 3561, July 2003.

\end{thebibliography}

\end{document}
